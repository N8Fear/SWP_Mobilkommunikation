\chapter{Evaluation}
Wir haben mit Hilfe des gegebenen tracking-Frameworks verschiedene Konstellationen mit einander verglichen:
\begin{itemize}
	\item[a)] die vorimplementierte Lösung, die mit Histogrammen arbeitet
	\item[b)] unsere HMM-basierte Lösung mit generischen Werten für Initialwerte, Emission und Transition ohne Lernen
	\item[c)] unsere HMM-basierte Lösung mit generischen Werten für Initialwerte, Emission und Transition mit Lernen
	\item[d)] unsere HMM-basierte Lösung mit vorgelernten Werten und Lernen
\end{itemize}
\section{Vergleich mit Histogramm-basierter Implementierung: Innenhof}

\section{Vergleich mit Histogramm-basierter Implementierung: Eingang}

\section{Vergleich mit Histogramm-basierter Implementierung: Tegel}

\section{Diskussion}
 %bezug auf anforderungen nehmen, punkte erfüllt?
 %hmm topologie?
 %andere libs?
 %forward vs. viterbi
 %dynamische histrogramm/cluster grenzen für dc wert in zukunft?
 %viterbi/forawrd gibt zustand, man kann entweder gewichtestes würfeln oder aber deterministisch die wahrschienlichste beobachtung nehmen, was ist besser? -> test?
 %performance gewinn durch threading zusätzlich
 
