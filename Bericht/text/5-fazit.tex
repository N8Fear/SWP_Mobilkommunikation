\section{Fazit und Ausblick}
\label{chap:fazit}


\subsection{CvHMM}
\label{sec:CvHMM}
Leider hat sich im Laufe des Projekts gezeigt, dass CvHMM leider sehr ineffizient auf einzelne Daten zugreift.\\

Ein weiterer großer Nachteil von CvHMM ist, dass nur der Baum-Welch-, der Viterbi- und der Decode-Algorithmus angeboten werden.
 Gerade hier ist wiederum ein Manko zu sehen, da wir später den Forward-Algorithmus gebraucht hätten und nun stattdessen den Viterbi-Algorithmus verwenden müssen, der jedoch ineffizienter als der Forward-Algorithmus ist.

Zusammenfassend lässt sich zur Wahl der HMM Bibliothek sagen, dass es wohl das Ergebnis des Projekts erheblich hätte verbessern können, wenn unsere Wahl auf eine andere Bibliothek gefallen wäre.

Mögliche Kandidaten zum Testen wären dabei HMMlib oder vielleicht sogar die, wie wir später herausgefunden haben, in OpenCV enthaltenen HMM-Funktionen, die allerdings innerhalb des OpenCV-Projekts nicht weiterentwickelt werden und derzeit des Status “deprecated” haben.



