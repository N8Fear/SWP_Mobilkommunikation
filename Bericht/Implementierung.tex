\chapter{Implementierung und praktische Details}

Dieses Kapitel soll einen kurzen Überblick zu den von uns verwendeten Technologien bieten, die wir zur Realisierung des theoretischen Entwurfs aus Kapitel 2 verwendet haben.

\section{Videomaterial}
Das uns zur Verfügung gestellte Bildmaterial besteht aus einzelnen Videosequenzen, wir können also nicht auf Live-Material arbeiten.
In der Praxis spielt dies jedoch keine große Rolle, da unter Linux der Zugriff auf eine Videodatei sich nicht wesentlich von dem Zugriff auf eine Kamera unterscheidet.
Insgesamt standen uns sieben unterschiedliche Videosequenzen zur Verfügung, von denen sechs paarweise entstandene Aufnahmen sind. Das bedeutet, dass Kameraposition sowie Winkel zwischen den beiden Videosequenzen eines Paares nicht differiert, lediglich die aufgenommenen Szenen sind unterschiedlich. Diese Tatsache ist insofern hilfreich, als das wir die Möglichkeit haben, auf einer Videosequenz zu lernen und die erlernten Parameter später auf der anderen Videosequenz anzuwenden.
Der größte Nachteil der Verwendung von Videosequenzen ist, dass die einzelnen Sequenzen relativ kurz sind (Dauer übersteigt nicht 10 Minuten) und es somit nicht möglich war, wirklich lange Lernphasen von zum Beispiel einigen Stunden unter realistischen Bedingungen zu testen.
Sämtliche Videosequenzen laufen mit 25 Frames pro Sekunde, was auf jeden Fall genug Daten zur Auswertung liefert. In der Praxis würden vermutlich auch schon weniger Frames genügen, da zu erwarten ist, dass sich die Personen im Bild nicht so schnell bewegen, dass sich innerhalb eines anderen Bruchteils einer Sekunde sehr viel verändert.

\section{C++}
Die Implementierung sowohl unseres Testbeds als auch die finale Implementierung als Komponente zur Foreground extraction in einem vorgegebenen Framework fand in C++ statt. Für die Wahl von C++ gab es verschiedene Gründe, von denen vor allem die Performance, die Portabilität sowie das gute Angebot an zur Verfügung stehenden Bibliotheken im Vordergrund standen. 

\section{OpenCV}
OpenCV ist eine Bibliothek die eine Vielzahl unterschiedlicher Algorithmen für die Bildbearbeitung und somit letztlich für die Videobearbeitung zur Verfügung stellt. Hinzu kommt, dass auch Funktionen für das Lesen, Schreiben und Abspielen von Videodateien angeboten werden.
OpenCV bietet Interfaces für C, Python, Java und C++ und lässt sich auf vielen unterschiedlichen Plattformen einsetzen. Für unser Projekt relevant waren dabei vor allem die Funktionen zur Wiedergabe von Videos sowie die in OpenCV enthaltene DCT Implementierung.

\section{CvHMM}
Die Wahl einer geeigneten Bibliothek für HMMs ist nicht trivial. Es existieren nicht sehr viele effiziente und aktiv entwickelte Bibliotheken für HMMs, die für unser Projekt in Frage kämen.
Letztlich fiel unserer Wahl auf CvHMM, vor allem da diese Bibliothek direkt auf dem in OpenCV enthaltenen Standard-Videodatentyp Mat arbeitet und daher die Vermutung nahe liegt, dass weniger Performanceeinbußen vorliegen, da keine Konvertierung von Daten in andere Formate erfolgen muss.
Leider hat sich im Laufe des Projekts gezeigt, dass CvHMM leider sehr ineffizient auf einzelne Daten zugreift.
Ein weiterer großer Nachteil von CvHMM ist, dass nur der Baum-Welch-, der Viterbi- und der Decode-Algorithmus angeboten werden. Gerade hier ist wiederum ein Manko zu sehen, da wir später den Forward-Algorithmus gebraucht hätten und nun stattdessen den Viterbi-Algorithmus verwenden müssen, der jedoch ineffizienter als der Forward-Algorithmus ist.
Zusammenfassend lässt sich zur Wahl der HMM Bibliothek sagen, dass es wohl das Ergebnis des Projekts erheblich hätte verbessern können, wenn unsere Wahl auf eine andere Bibliothek gefallen wäre.
Mögliche Kandidaten zum Testen wären dabei HMMlib oder vielleicht sogar die, wie wir später herausgefunden haben, in OpenCV enthaltenen HMM-Funktionen, die allerdings innerhalb des OpenCV-Projekts nicht weiterentwickelt werden und derzeit des Status “deprecated” haben.


